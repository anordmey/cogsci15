\documentclass[10pt,letterpaper]{article}

\usepackage{hyperref}
\usepackage{cogsci}
\usepackage{pslatex}
\usepackage{pdfsync}
\usepackage{apacite}
\usepackage{amsmath}
\usepackage{graphicx}
\usepackage{topcapt}
\usepackage{color}
\usepackage[textsize=small,textwidth=1.8cm]{todonotes}

\setlength{\marginparwidth}{1.5cm}

\title{The pragmatics of negation across contexts}
 % (title-in-progress:) Felicity ratings of negative sentences in context

\author{{\large \bf Ann E. Nordmeyer} \\ \texttt{anordmey@stanford.edu}\\ Department of Psychology \\ Stanford University \\ 
\And {\large \bf Michael C. Frank} \\ \texttt{mcfrank@stanford.edu} \\ Department of Psychology \\ Stanford University \\ }

\begin{document}

\maketitle


\begin{abstract}
Previous work \todo{Don't start with ``previous work'' - give us the question!} has shown that context affects negation processing \cite{wason1965, nordmeyer2014}. We explore the pragmatics of negation by examining adults' explicit felicity judgments of negative sentences in context.  In Experiment 1, we found that negative sentences presented within a pragmatically strong context were rated as more felicitous, and that negative sentences expressing nonexistence were rated higher than negative sentences referring to an alternative object.  In Experiment 2, we found that negative sentences referring to an alternative object were rated higher when they were presented in a context where all of the other characters possessed the negated object, compared to contexts where the other characters possessed the alternative object or no object.  We discuss the pragmatics of negation in light of these results, arguing that the felicity of negative sentences is influenced by changes in the informativeness of these sentences in different contexts. 

\textbf{Keywords:} 
Negation; felicity judgments; pragmatics
\end{abstract}

\section{Introduction}

A sentence that is both grammatical and true can nevertheless sound odd in some contexts. If you open a mysterious box and find that it is empty, it's weird to say ``This box doesn't have any chocolates in it,'' despite the truth of the proposition. But the same sentence becomes perfectly reasonable if you are in a chocolate store surrounded by boxes filled with chocolate.  Why do some sentences sound strange (\emph{infelicitous}) in one situation, but normal in another?

Theories of pragmatics attempt to provide an account of how language users move from the literal semantics of a sentence to an inference about the speaker's intended meaning. For example, according to Grice's \citeyear{grice1975} Cooperative Principle, speakers should produce utterances that are truthful, relevant, and informative.  By assuming that speakers do so, listeners can make inferences about intended meaning that go beyond the sentence's literal meaning. Modern neo-Gricean theories tend to derive such inferences from the tension between being informative with respect to a communicative goal and minimizing the effort expended \cite{horn1984,levinson2000,frank2012}. These theories make predictions about sentence felicity: As in our example above, a sentence that is not pragmatically optimal can sound strange even when it is both grammatical and true.

We explore this relationship between contextual pragmatics and felicity using negation. As the previous example demonstrates, negative sentences are particularly sensitive to to effects of context, making them a good case study.  When they are presented without any context, negative sentences are difficult to process relative to positive sentences \cite{hclark1972, carpenter1975, just1971, just1976}, but supportive contexts can substantially reduce this processing difficulty \cite{wason1965, glenberg1999, ludtke2006, nieuwland2008, dale2011, nordmeyer2014b}.   Previous work has linked reaction time to syntactic \emph{surprisal}, an information-theoretic measure predicting that low-probability utterances will be processed slower than high-probability utterances \cite{levy2008}.  If \emph{pragmatic surprisal}---the probability of an utterance being produced given the context---is responsible for the context effects seen in negation processing, these context effects should be reflected in adults' explicit felicity judgments as well.   %<-- does the end of this paragraph work/make sense??

%The context of a negative sentence can change the type of negation.  
%Most of the previous work on adults' comprehension of negative sentences has focused on denial, or truth-functional negation -- that is, negative sentences expressing that some proposition is false.  This is not the only function of negation, however; negative sentences can be used to express nonexistence, refusal, and prohibition, among many others \cite<see>{choi1988}.  
We focus on two types of negation: \emph{nonexistence} and \emph{alternative} negation.  The same sentence can express either of these concepts depending on the context.  Referring to our previous example, the sentence ``This box doesn't contain chocolate'' might refer to an empty box (e.g. nonexistence) or a box containing some alternative object (e.g. broccoli instead of chocolate).  Previous work suggests that adults are faster to identify the referent of nonexistence negation compared with alternative negation \cite{nordmeyer2013, nordmeyer2014b}, but it is not clear if this is due to processing demands (e.g. identifying a missing feature is superficially easier than identifying a changed feature), or if this reflects an expectation about the pragmatics of negation.  Adults may find alternative negation infelicitous  because they expect the sentence to describe features that a character \emph{has}, rather than a feature that they do \emph{not} have.


%  In previous work, we demonstrated that negative sentences are more informative in contexts that set up a strong expectation that is violated, and that this difference in informativeness can predict adults' reaction times to respond to negative vs. positive sentences \cite{nordmeyer2014}.  

We examined the effects of context on adults' explicit felicity judgments for different negative sentences.  In Study 1, we tested the effect of context on nonexistence negation and alternative negation.  We found that negative sentences presented in a strong context were rated as more felicitous, and that nonexistence negation was rated higher than alternative negation.  In Study 2, we focused on alternative negation and found that alternative negation is rated as more felicitous in a context where all of the other characters possessed the negated object, compared to contexts where the other characters possessed the alternative object or no object.  We discuss these results within a neo-Gricean framework, demonstrating that a model of informativeness can predict the same qualitative pattern seen in our data.  This suggests that adults' felicity judgments reflect preferences for negative sentences that are more informative given the context.  

\section{Experiment 1}

Experiment 1 explored how different contexts affected participants' felicity ratings for negative sentences.  Half of the participants in Experiment 1 saw sentences presented in a context where none of the characters had any objects, and the other half saw sentences presented in a context where everyone except for the target character possessed the negated object.  In half of the true negative trials, the referent of the negative sentence was a character who had nothing (e.g. nonexistence negation).  In the other half of true negative trials, the referent of the sentence was a character who had some other object (e.g. alternative negation).  
%Previous work suggests that children and adults show different gaze patterns in response to these different types of negative sentences \cite{nordmeyer2014b}; here we explore whether adults' have a preference for the type of negation expressed in different negative sentences, and whether both types of negation are effected by context.
Finally, we examined how different syntactic framings might influence sentence judgments (``has no X'' and ``doesn't have X''). 
% In our previous work on the effects of context on reaction time, we tested sentences such as ``X has no Y.''  In this experiment, we also tested sentential negation, using sentences such as ``X doesn't have Y.''  We were interested in whether participants had a preference for one framing over another, and whether this would interact with context.  
If previously seen context effects are driven by the informativeness of negation in different contexts, then these same effects should appear regardless of the syntactic framing of the sentences.  

\subsection{Method}

\subsubsection{Participants}

We recruited 94 adults to participate in an online experiment through Amazon Mechanical Turk.  Participants ranged in age from 18-65; 50 were male and 41 female (three declined to report gender).  We restricted participation to individuals in the United States. We paid 35 cents for the experiment, which took approximately five minutes to complete.  

\subsubsection{Stimuli}

We created 16 trial items. On each trial, four Sesame Street characters were shown standing behind tables.  One character was randomly selected as the ``target'' character, designated by a red box around that character and their table. The remaining three characters were designated as ``context'' characters.


\begin{figure}[t]
\begin{center} 
\includegraphics[width=3in]{figures/trialtypes.pdf}
\caption{\label{fig:trial} The five types of true negative trials across Experiment 1 and Experiment 2.  In the experiments, Sesame Street characters appeared behind each table.  Below each set of pictures a sentence appeared about the character in red, e.g. ``X doesn't have apples.''}.  
%I took out the sesame street characters because it felt way too busy with them in -- does that seem OK?
\end{center} 
\end{figure}

Participants were randomly assigned to the ``none context'' condition or the ``target context'' condition.  In the none context condition, the context characters all stood behind empty tables.  In the target context condition, each context character had an identical object on their tables.  The objects belonged to one of four categories: animals (cat, dog, horse cow), vehicles (car, bus, boat, truck), food (apple, banana, cookie, orange), and household objects (fork, spoon, bowl, plate).  

Below the characters was a sentence about the target character.  Six of these sentences were positive (``[character] has a/an [object].''  Five were negative sentences of the form ``[character] has no [object]'' and five were negative sentences of the form ``[character] doesn't have a/an [object]''.  

The target character either had a target object on their table, or an alternative object (``alternative'' trials), or nothing (``nonexistence'' trials).  This allowed us to examine two different negative concepts (see Figure \ref{fig:trial} \todo{This is great, but the figure/sentence correspondence doesn't make total sense when no character is present...} for depictions of these trials and the different context conditions).    These trial conditions were crossed such that each participant saw six true positive trials, two false positive trials (one alternative and one nonexistence), two false negative trials (one ``has no'' sentence type and one ``doesn't have'' sentence type), and eight true negative trials (two ``has no''/nonexistence, two ``has no''/alternative, two ``doesn't have/nonexistence'', and two ``doesn't have/alternative'').  Each of these trial types was randomly assigned to a target object, and trials were presented in a random order.  

A slider bar was positioned beneath the sentence, with a seven-point scale ranging from ``Very Bad'' to ``Very Good.''  A progress bar at the top of the screen informed participants how much of the experiment they had completed. 

\subsubsection{Procedure}

Participants were first presented with an instructions screen that briefly described the task and informed them that they could stop at any time.  Once participants agreed to participate in the task, they saw an instructions screen that explained the task in more detail.  The instructions explained that participants would see a sentence about the character in the red square.  Participants were told that their job was to rate how ``good'' each sentence is, and ``if no one would ever say a particular sentence in this context, or if it is just wrong, rank that as `Very Bad,' but if something is right and sounds perfectly normal, mark it as `Very Good.' ''  Participants were encouraged to use the entire scale. On each trial the pictures, sentence, and slider bar appeared simultaneously, and participants had to make a selection on the sliding scale in order to progress to the next trial.  


\subsubsection{Data Processing}

We excluded from analysis two participants who did not list English as their native language, and eight participants for using fewer than three points on the scale.  Thus, data from a total of 84 participants were analyzed, 46 in the none context condition and 38 in the target context condition.  

\subsection{Results and Discussion}

%\subsubsection{Effects of context}

True negative sentences were rated significantly higher in a target context compared to a none context (Figure \ref{fig:s1}). For example, the sentence ``Abby has no apples'' was rated higher when all of the other characters \emph{had} apples, compared to contexts where all of the other characters had nothing. This finding supports our hypotheses and replicates previous processing-time effects using explicit felicity judgments.  Negative sentences expressing nonexistence were rated as more felicitous than negative sentences referring to an alternative object, and sentences with the framing ``doesn't have'' were rated higher than sentences with the framing ``has no'' (see Figure \ref{fig:s1}). True positive sentences did not show any effect of context, nor did false sentences of any sentence type, likely due to a ceiling effect for true positive sentences and a floor effect for false sentences.  

%We ran a linear mixed-effects model testing the interaction between context, sentence type, and truth value on participant sentence ratings.\footnote{All mixed-effects models were fit using the lme4 package in R version 2.15.3.  The model specification was as follows: \texttt{rating $\sim$ context~$\times$~sentence~$\times$~truth + (sentence~$\times$~truth~\textbar~subject) +  (sentence~$\times$~truth~\textbar~item)}  Significance was calculated using the standard normal approximation to the $t$ distribution \cite{barr2013}. Data and analysis code can be found at \href{http://github.com/anordmey/cogsci15}{http://github.com/anordmey/cogsci15}}.  Results of this model showed a main effect of sentence type, with negative sentences receiving significantly lower ratings than positive sentences ($\beta= -.58$, $p< .01$), and a main effect of truth value, with false sentences scoring approximately four points lower on the 7-point scale compared to true sentences ($\beta= 4.19$, $p< .001$).  There was not a main effect of context ($\beta= -0.52$, $p=.12$), but there was a significant interaction between sentence type and context, with negative sentences scoring significantly higher when presented in a strong context compared to a weak context ($\beta= 2.04$, $p< .05$).  

%\subsubsection{Effects of negation type and sentence type}

 
To test the reliability of these findings, we fit a linear mixed-effects model examining the interaction between context, negation type (e.g. nonexistence or alternative), and negation framing (e.g. ``has no X'' vs ``doesn't have X'') on true negative sentence ratings.\footnote{ The model specification was as follows: \texttt{rating $\sim$ context~$\times$~negation type~$\times$~negation type + (negation type~$\times$~negation frame~\textbar~subject) +  (negation type~$\times$~negation frame~\textbar~item)}.  Significance was calculated using the standard normal approximation to the $t$ distribution \cite{barr2013}. Data and analysis code can be found at \href{http://github.com/anordmey/cogsci15}{http://github.com/anordmey/cogsci15}}.  We focused here on responses to true negative sentences only in order to explore the interaction between these three factors, which were only relevant for true negative trials.  The effects of context reported here are significant in a full model of all sentence types as well. We found a main effect of context, with true negative sentences presented in a strong context eliciting significantly higher ratings than true negative sentences presented in a weak context ($\beta= .99$, $p< .001$).  We also found main effects of negation type, with sentences referring to an alternative receiving lower ratings than sentences expressing nonexistence ($\beta= -.46$, $p< .05$), as well as negation framing, with sentences of the form ``has no X'' receiving lower ratings than sentences of the form ``doesn't have X''  ($\beta= -.46$, $p< .05$).  There were no interactions between negation frame, negation type, and context.  

The difference between the two negation sentence frames is likely due to frequency effects in the input. \todo{GET SOME DATA!}

Our previous eye-tracking experiment found that adults were somewhat faster to look at the referent of nonexistence negation, compared with alternative negation \cite{nordmeyer2014b}. In that experiment, however, this difference could have arisen because of superficial, stimulus-level differences, since it might be easier to identify a character with nothing than one with an alternative object. Our replication of this result here using explicit felicity judgment suggests that this difference in eye-movements might have resulted from pragmatic factors as well. One explanation for participants' preference for nonexistence negation is that a negative sentence such as ``Abby doesn't have an apple'' is more informative when Abby has nothing compared to when she has an alternative object.  In a strong context (e.g. one where everyone besides Abby has apples), using negation to point out Abby's lack of apples is informative, because it uniquely identifies her character in the array.  When Abby has some alternative object (e.g. a cat), however, there is a \emph{more} informative utterance that a speaker could use (e.g. ``Abby has a cat'').  The presence of a more informative utterance makes these negative sentences less felicitous, even when they appear in context.

Overall, we found that a pragmatic context increases the felicity judgments of negative sentences.  Across all types of negative sentences, participants assigned higher ratings to sentences that were presented with a strong target context compared to sentences that were presented with a none context.  This supports findings from previous work, in which negative sentences that were more informative based on the context were processed faster than less informative negative sentences \cite{nordmeyer2014}.  In the next section, we expand on this finding by testing the same sentences in a new context to further explore the pragmatics of negative sentences.

\begin{figure}
\begin{center} 
\includegraphics[width=3.25in]{figures/study1.pdf}
\caption{\label{fig:s1} Ratings for different types of true negative sentences in Experiment 1.  Sentences of the form ``...has no X'' are shown on the left, and sentences of the form ``...doesn't have X'' are shown on the right.  Negative sentences expressing nonexistence are shown in black, and negative sentences referring to an alternative object are shown in gray.  Error bars show 95\% confidence intervals.}
\end{center} 
\end{figure}

\begin{table}[t]
\caption{\label{tab:s1} Coefficient estimates from a mixed-effects model predicting ratings for true negative sentences in Experiment 1.}
\begin{center}
\small\addtolength{\tabcolsep}{-5pt}
\begin{tabular}{rrrr}
  \hline
 & Coefficient & Std. err. & t value \\ 
  \hline
(Intercept) & 5.24 & 0.20 & 25.94 \\ 
  Context (Context) & 0.99 & 0.29 & 3.46  \\ 
  Negation type (alternative) & -0.46 & 0.18 & -2.51 \\
  Frame (``has no'') & -0.51 & 0.25 & -2.01 \\ 
  Context $\times$Negation type & -0.32 & 0.24 & -1.32 \\
  Context $\times$Frame & -0.18 & 0.35 & -0.50 \\
  Negation type$\times$Frame & -0.19 & 0.22 & -0.83 \\
  Context$\times$Negation type$\times$Frame & 0.20 & 0.33 & 0.62 \\
   \hline
\end{tabular}
%\vspace{-1.5cm}
\end{center}
\end{table}

\section{Experiment 2}

%Experiment 1 found clear differences in participants' ratings of negative sentences.  Participants preferred negative sentences with the sentential framing ``doesn't have X'' compared to the lexical framing ``has no X.''  Participants also preferred negative sentences that referred to nonexistence.  Across both of these factors, negative sentences that were presented within a strong context (where all context characters possessed the negated object) received higher ratings than negative sentences presented in a weak context, where none of the characters in the context possessed any object.

In Experiment 2, we focused specifically on alternative negation sentences with a ``doesn't have'' framing to further explore the effects of context.  We created a within-subjects context factor, so that participants saw multiple items in different types of contexts throughout the experiment.  We examined the effect of three contexts: a target context, in which all context characters had the negated target object (identical to Experiment 1), a none context, in which none of the context characters had any objects (identical to Experiment 1), and a foil context, in which all context characters had an alternative object (e.g. a different object than the one negated in the negative sentences). 

We expected to replicate the same difference between the none context and the target context as was seen in Experiment 1: Negative sentences presented in a target context should receive higher ratings than negative sentences presented in a none context.  We had two competing predictions about the foil context.
% ((NOTE: I actually didn't articulate this until after running the experiments, but I think it works as a possible alternative hypothesis.  Is it OK to set this up like this))).  
In true negative trials with a foil context, all characters (including the target character) had the same objects on their table (e.g. cats), and the negative sentence referred to a different object (e.g. ``X doesn't have apples'').  Some previous work suggests that a critical element of the effect of context on negative sentences is the fact that the referent of the negative sentence is the ``odd one out'' \cite{wason1965}.  If this is the case, the foil context might be even worse than the none context, because the target of the negative sentence does not stand out from the context.  If, however, context changes the informativeness of negation, then there should be no difference between the foil and none contexts, because negative sentences are no less informative in the foil context.  

\subsection{Method}

\subsubsection{Participants}

We recruited 194 participants to participate in an online experiment through Amazon Mechanical Turk.  Participants ranged in age from 18-65; 115 were male and 76 female (three declined to report gender).  We restricted participation to individuals in the United States. We paid participants 40 cents and the experiment took approximately seven minutes to complete.  

\subsubsection{Stimuli}

Trials in Experiment 2 had the same structure as trials in Experiment 1, with a small set of exceptions. First, there were 24 trials. Second, all negative sentences were of the form ``[charager] doesn't have a/an [object].'' Third, on each trial, the target character either had a target item on their table, or had an alternative item (eliminating the nonexistence trials).  Finally, each participant evaluated nine true positive sentences, three false positive, three false negative, and nine true negative trials.

In Experiment 2, context was a within-subjects factor with three levels. In the none context, context characters had nothing on their table, identical to the none context condition in Experiment. In the target context, context characters each had a target item on their table, same as the target context condition in Experiment 1. In the foil context, context characters had an alternative item on their table (e.g., all characters have cat but the sentence is about the presence/absence of apples; see Figure \ref{fig:trial}).  Each context condition appeared an equal number of times within each trial type.  

\subsubsection{Procedure}

The procedure was identical to Experiment 1.

\subsubsection{Data Processing}

We excluded from our analysis four participants who did not list English as their native language, six participants for having participated in a previous pilot study, and 34 participants for using fewer than three points on the scale.  Thus, data from a total of 154 participants were analyzed.  

\subsection{Results and Discussion}

True negative sentences were rated significantly higher when they were presented in a target context compared to either the none context or the foil context (Figure \ref{fig:s2}).  Negative sentences presented in a foil context received a very small increase in ratings compared to negative sentences presented in a none context. As in Experiment 1, there was no effect of context on true positive sentences or false sentences.  

% ((NOTE: I'm in a weird position here where it REALLY does not look like there is a significant difference here (I actually initially wrote that there WASN'T a difference), but it does show up as marginally significant in my data (and actually significant if I run a model on just the true sentences, which I don't report here but did on my own).  So I feel weird describing this as an actual difference in this first paragraph, but it does show up in my data -- and also in my model -- so I feel like I should?))).  

\begin{figure}
\begin{center} 
\includegraphics[width=3.25in]{figures/study2.pdf}
\caption{\label{fig:s2} Ratings for true negative sentences in three context conditions in Experiment 2.  Error bars show 95\% confidence intervals.}
\end{center} 
\end{figure}

We fit a linear mixed-effects model to sentence ratings to test the interaction between context, sentence type, and truth value (Table \ref{tab:s2}).\footnote{ The model specification was as follows: \texttt{rating $\sim$ context~$\times$~sentence type~$\times$~truth value + (sentence type~$\times$~truth value~\textbar~subject) +  (sentence type~$\times$~truth value~\textbar~item)}}  We found a significant effect of truth value, with true sentences receiving significantly higher ratings than false sentences, ($\beta= 5.01$, $p< .001$).  A significant interaction between sentence type and truth value indicates that true negative sentences received lower ratings than true positive sentences, ($\beta= -1.78$, $p< .001$).  Finally, a three-way interaction between between sentence type, truth value, and context was significant for the target context ($\beta= 1.08$, $p< .001$), and marginally significant for the foil context ($\beta= .34$, $p=.06$), indicating that true negative sentences received significantly higher ratings in the target context compared to the none context, and slightly higher ratings in the foil context compared to the none context.  

\begin{table}[t]
\caption{\label{tab:s2} Coefficient estimates from a mixed-effects model predicting sentence ratings in Experiment 2.}
\begin{center}
\small\addtolength{\tabcolsep}{-5pt}
\begin{tabular}{rrrr}
  \hline
 & Coefficient & Std. err. & t value \\ 
  \hline
(Intercept) & 1.62 & 0.11 & 15.07 \\ 
  Context (foil) & 0.13 & 0.11 & 1.14  \\ 
  Context (target) & 0.05 & 0.11 & 0.43  \\ 
  Sentence type (negative) & 0.11 & 0.12 & 0.94 \\
  Truth value (true) & 5.01 & 0.13 & 37.60 \\ 
  Context (foil)$\times$Sentence & -0.16 & 0.16 & -0.98 \\
  Context (target)$\times$Sentence & -0.11 & 0.16 & -0.72 \\
  Context (foil)$\times$Truth & -0.19 & 0.13 & -1.48 \\
  Context (target)$\times$Truth & -0.14 & 0.13 & -1.08 \\
  Sentence$\times$Truth & -1.78 & 0.16 & -11.40 \\
  Context (foil)$\times$Sentence$\times$Truth& 0.34 & 0.18 & 1.87 \\
  Context (target)$\times$Sentence$\times$Truth & 1.08 & 0.18 & 5.88 \\
   \hline
\end{tabular}
\end{center}
\end{table}

In Experiment 2, negative sentences presented in a context where all of the context characters possessed the negated object were rated as the most felicitous.  Negative sentences presented in a foil context, where all characters had the same alternative object, elicited marginally higher ratings than negative sentences in a none context.  This rules out the possibility that negative sentences are more pragmatically valid in contexts where the negated object is the ``odd one out.''  Instead, negation appears to be pragmatically licensed in contexts where the negative sentence is highly informative in the context.  

\section{Model}

% Studies 1 and 2 explored the effects of context on different types of negative sentences.  
Both of the preceding studies found a significant effect of context on participants' ratings of true negative sentences.  Why does context have this effect on negative sentences?  One hypothesis is that felicity ratings are influenced by the \emph{informativeness} of negative sentences. On theories of pragmatics, speakers should produce sentences that are appropriately informative based on the context, and in a context where most characters have apples and another character has a cat, it is informative to mention the ``odd character out'' by describing the cat, because this feature is unique to the character being described.

% Previous work suggests that a probabilistic formalization of Gricean pragmatics \cite{frank2012} could predict adults' speed to evaluate negative sentences \cite{nordmeyer2014a}.  Here, we use the same model to explore whether participants' felicity judgments reflect a preference for sentences that are more informative based on the context.  


% We focus on true negative sentences here, because the ceiling effect for true positive sentences and the floor effect for negative sentences make it difficult to consider the effect of context on these sentences.  

We used a recent probabilistic model of pragmatics to make predictions about participants' felicity ratings, testing this hypothesis. Details of this model---the ``rational speech act'' model of pragmatics---can be found in a number of previous publications \cite{frank2012,goodman2013,nordmeyer2014}. Here we give a brief sketch of the intuitions behind the model. The probability of a speaker making an utterance in context is defined as being  proportional to the informativeness of the word in context minus its cost.  Informativeness in context is calculated as the number of bits of information conveyed by the utterance.  We assume that the utterance has a uniform probability distribution over its extension in context (e.g., ``doesn't have apples'' applies to any character who does not have apples, leading to a uniform probability of picking out each individual character with apples). We defined cost as the number of words in the utterance multiplied by a cost-per-word parameter; in our simulations, we did not differentiate between different negative sentence frames, and treat negative sentences as having one word more than positive sentences.  Probabilities were normalized over a sparse vocabulary of possible positive or negative utterances that could be used to describe the characters.

\begin{figure}[t]
\begin{center} 
\includegraphics[width=3.25in]{figures/model_predictions.pdf}
\caption{\label{fig:model} Predictions of a model of the informativeness of an utterance in context.  The model predicts how probable a sentence is given a certain context.  Negative sentences expressing nonexistence are shown in black, and negative sentences referring to an alternative object are shown in gray.}
\end{center} 
\end{figure}

Our model predicted the same qualitative pattern as the data from participants' sentence ratings (Figure \ref{fig:model}).  Sentences presented in a target context were preferred over sentences presented in a none context, because true negative utterances are more informative when everyone in the context possesses the negated object (e.g., the sentence ``Abby doesn't have apples'' uniquely identifies Abby when every other character \emph{does} have apples).  Nonexistence negation was assigned higher probability than alternative negation, because alternative negation could be described by an equally informative---and less costly---positive utterance (e.g., ``Abby has cats'' instead of ``Abby doesn't have apples'').  Thus, participants' sentence ratings appear to parallel the informativeness of sentences in context.  

\section{General Discussion}

% What is the effect of context on negation?  
The same negative sentence can sound perfectly fine in one context, but strange in another.  It can refer to \emph{nothing} in one context, but a \emph{difference} in another. In our experiments, we found that contextual differences led to significantly different pragmatic judgments for otherwise identical true, grammatical negative sentences. What is it about the context of negative sentences that elicits these effects?  

We hypothesized that it was negative sentences that were more informative---more effective at conveying the intended referent of the sentence---that were rated as more felicitous.  This hypothesis was supported across multiple contexts, including different types of negation.
 % Contexts where the referent of negation was nonexistence were rated higher than contexts where the referent of negation was an alternative.  
Our findings are consistent with the relative felicity of different negation being determined by general pragmatic factors, under a model in which the goal of communication is to convey your intended meaning in the most efficient and effective way possible.  

Grice's Cooperative Principle suggests that speakers should produce utterances that are maximally informative given the context, in order to effectively communicate their intentions to a listener.  If listeners expect speakers to abide by this principle, they should prefer sentences that are more informative.  Our results support this view of communication: Sentences that were more informative (and therefore more likely to be produced by a speaker) were given higher felicity ratings than less informative sentences.  Negative sentences typically provide the listener with very little information; in general, a positive utterance is more informative.  This is particularly true in the case of alternative negation, where an informative positive utterance is readily available.  A negative utterance can become informative, however, in a context where some salient violation of expectations has occurred (e.g. the target context in our experiments).  In these contexts, not only does a listener accept a negative utterance about this violation of expectations, but they may even \emph{expect} a speaker to produce this utterance.

% Previous work on children's acquisition of negation has found that preschoolers struggle to respond correctly to true negative sentences \cite{kim1985, nordmeyer2014b}, despite producing negative sentences spontaneously (and accurately) before they are two years old \cite{pea1980, pea1982}.  One possibility is that children are behaving rationally according to a Gricean view of communication.  In most experiments of children's comprehension of negation, children hear negative sentences without any supportive pragmatic context.  In contrast, studies that have elicited spontaneous negations from children tend to use familiar contexts, such as reading picture books in an interactive, game-like setting \cite{pea1982, hummer1993}.  When children hear a true negative sentence and indicate that it is ``wrong'' \cite<e.g.>{kim1985}, they may be reacting to the infelicity of the sentence rather than its truth value.  



% Future work will explore children's comprehension of negative sentences in context as a case study to explore children's sensitivity to pragmatics and context.  %Is this last paragraph worth including?  If I include it, should I say something after this?


\section{Acknowledgments}

The first author gratefully acknowledges the support of a National Science Foundation Graduate Research Fellowship. 

\bibliographystyle{apacite}

\setlength{\bibleftmargin}{.125in}
\setlength{\bibindent}{-\bibleftmargin}

\bibliography{negation}


\end{document}
