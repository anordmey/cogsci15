\documentclass[10pt,letterpaper]{article}

\usepackage{hyperref}
\usepackage{cogsci}
\usepackage{pslatex}
\usepackage{pdfsync}
\usepackage{apacite}
\usepackage{amsmath}
\usepackage{graphicx}
\usepackage{topcapt}
\usepackage{color}


\title{Something about felicity judgments and negative sentences and context}
 
\author{{\large \bf Ann E. Nordmeyer} \\ \texttt{anordmey@stanford.edu}\\ Department of Psychology \\ Stanford University \\ 
\And {\large \bf Michael C. Frank} \\ \texttt{mcfrank@stanford.edu} \\ Department of Psychology \\ Stanford University \\ }

\begin{document}

\maketitle


\begin{abstract}
Blah blah blah felicity judgments and negative sentences and context

\textbf{Keywords:} 
Negation; felicity judgments; pragmatics
\end{abstract}

\section{Introduction}

Experiment 1 shows that context effect persists across different uses/types of negation.

Experiment 2 shows that particular context is effective (setting up an expectation that is violated).

Experiment 3: presence/absence of target item doesn't have an effect.   

Punchy intro paragraph.


Talk about negation \& why it is interesting/important.

Talk about the role of context in processing negation.

Talk about our thoughts about why context has this effect.

Talk about what we are missing: explicit ratings from people.  What do we gain from this?  We can 1) get clearer ratings than we can get with a noisy measure like reaction time, and 2) evidence that this is an explicit preference that people have (for negative sentences in supportive contexts) and not simply an implicit bias (???)

\section{Study 1}

Study 1 explored the effects of multiple factors on the felicity judgments of negative sentences.  Half of the participants in Study 1 saw sentences presented within a strong context (e.g. one where all of the context characters had the negated object), and the other half saw sentences presented within a weak context (e.g. one where none of the context characters had any objects).  This manipulation has been shown to influence reaction times to evaluate negative sentences \cite{nordmeyer2014}; here we test whether context changes explicit felicity judgments as well.  We also tested two within-subjects factors that might influence negative sentence judgments.  First, we looked at how different syntactic framings might influence sentence judgments.  In our previous work on the effects of context on reaction time, we tested sentences such as ``X has no Y''.  In this experiment, we also tested sentential negation, using sentences such as ``X doesn't have Y''.  We were interested in whether participants had a preference for one framing over another, and whether this would interact with context.  If previously seen context effects are driven by the informativeness of negation in different contexts, then these same effects should appear regardless of the syntactic framing of the sentences.  Finally, we examined whether the type of negation influenced felicity ratings.  In half of the true negative trials, the referent of the negative sentence was a character who had nothing.  In these sentences, the negation refers to the \emph{nonexistence} of the named item.  In the other half of true negative trials, the referent of the sentence was a character who had some other object.  In these sentences, the negation refers to the fact that the character has an \emph{alternative} object.  Previous work suggests that children and adults show different gaze patterns in response to these different types of negative sentences \cite{nordmeyer2014b}; here we explore whether adults' have a preference for the type of negation expressed in different negative sentences.

\subsection{Method}

\subsubsection{Participants}
We recruited 94 participants to participate in an online experiment through the Amazon's Mechanical Turk (mTurk) website.  Participants ranged in age from 18-65; 50 male and 41 female (3 declined to report gender).  We restricted participation to individuals in the United States. We paid participants 35 cents to participate, which took approximately 5 minutes to complete.  


\subsubsection{Stimuli}
We created sixteen trial items.  On each trial, four different Sesame Street characters were shown standing behind tables.  One character was randomly selected as the "target" character, designated by a red box around that character \& the character's table, and the remaining three characters were designated as "context" characters.

Participants were randomly assigned to the ``no context'' condition or the ``context'' condition.  In the no context condition, the context characters all stood behind empty tables.  In the context condition, each context character had identical objects on their tables (``target objects'').  The objects belonged to one of four categories: animals (cat, dog, horse cow), vehicles (car, bus, boat, truck), food (apple, banana, cookie, orange), and household objects (fork, spoon, bowl, plate).  

Below the characters was a sentence about the target character.  Six of these sentences were positive sentences of the form ``[CHARACTER] has a/an [TARGET OBJECT].''  Five were negative sentences of the form ``[CHARACTER] has no [TARGET OBJECT]'' and five were negative sentences of the form ``[CHARACTER] doesn't have a/an [TARGET OBJECT]''.  

The target character either had a target object on their table, or an alternative object (``alternative'' trials), or nothing (``nonexistence'' trials).  This allowed us to examine two different negative concepts (negation-as-alternative and negation-as-nonexistence).  These trial conditions were crossed such that each participant saw six true positive trials, two false positive trials (one ``alternative'' and one ``nonexistence''), two false negative trials (one ``has no'' sentence type and one ``doesn't have'' sentence type), and eight true negative trials (two ``has no''/nonexistence, two ``has no''/alternative, two ``doesn't have/nonexistence'', and two ``doesn't have/alternative'').  Each of these trial types was randomly assigned to a target object, and trials were presented in a random order.

A slider bar was positioned beneath the sentence, with a 7 point scale ranging from ``Very Bad'' to ``Very Good''.  A progress bar at the top of the screen informed participants how much of the experiment they had completed. 

\begin{figure}[t]
\begin{center} 
\includegraphics[width=3.25in]{figures/example.png}
\caption{\label{fig:trial} A "true negative" trial with the alternative negation type.}
\vspace{-5mm}
\end{center} 
\end{figure}

\subsubsection{Procedure}
Participants were first presented with an instructions screen that briefly described the task and informed them that they could stop at any time.  Once participants had agreed to participate in the task, they saw an instructions screen that explained the task in more detail.  The instructions explained that participants would see a sentence, and that this sentence was about the character in the red square.  Participants were told that their job was to rate how ``good'' each sentence is, and ``if no one would ever say a particular sentence in this context, or if it is just wrong, rank that as `Very Bad', but if something is right and sounds perfectly normal, mark it as `Very Good'.''  Participants were encouraged to use the entire scale to make their ratings.

On each trial the pictures, sentence, and slider bar appeared simultaneously, and participants had to make a selection on the sliding scale in order to progress to the next trial.  

\subsubsection{Data Processing}
We excluded from analysis two participants who did not list English as their native language, and eight participants for using fewer than 3 points on the scale.  Thus, data from a total of 84 participants were analyzed, 46 in the no context condition and 38 in the context condition.  

\subsection{Results \& Discussion}

\subsubsection{All sentences}

True negative sentences were rated significantly higher when they were presented in a strong context compared to a weak context (see Figure \ref{fig:s1}), supporting our hypotheses and previous research on the effects of context on negation.  True positive sentences did not show any effect of context, nor did false sentences of any sentence type, likely due to a ceiling effect for true positive sentences and a floor effect for false sentences.  

We ran a linear mixed-effects model testing the interaction between context, sentence type, and truth value on participant sentence ratings.\footnote{All mixed-effects models were fit using the lme4 package in R version 2.15.3.  The model specification was as follows: \texttt{rating $\sim$ context~$\times$~sentence~$\times$~truth + (sentence~$\times$~truth~\textbar~subject) +  (sentence~$\times$~truth~\textbar~item)}  Significance was calculated using the standard normal approximation to the $t$ distribution \cite{barr2013}. Data and analysis code can be found at \href{http://github.com/anordmey/FILLTHISIN}{http://github.com/anordmey/FILLTHISIN}}.  Results of this model showed a main effect of sentence type, with negative sentences receiving significantly lower ratings than positive sentences ($\beta= -.58$, $p< .01$), and a main effect of truth value, with false sentences scoring approximately four points lower on the 7-point scale compared to true sentences ($\beta= 4.19$, $p< .001$).  There was not a main effect of context ($\beta= -0.52$, $p=.12$), but there was a significant interaction between sentence type and context, with negative sentences scoring significantly higher when presented in a strong context compared to a weak context ($\beta= 2.04$, $p< .05$).  

These results confirm that the effects of context seen in the sentence processing literature on reaction times to respond to positive and negative sentences are also present in participants' explicit felicity ratings.  Participants rated true negative sentences as better when the other characters in the context had the negated item.  For example, the sentence ``Abby has no apples'' was rated higher when all of the other characters \emph{had} apples, compared to contexts where all of the other characters had nothing.  

\subsubsection{True negative sentences}

Negative sentences expressing nonexistence were rated as more felicitous than negative sentences that referred to an alternative object, and sentences with the framing ``doesn't have'' were rated higher than sentences with the framing ``has no''.  These effects persisted regardless of whether the sentence was presented in a weak context or a strong context (see Figure \ref{fig:s1}).  

To test the reliability of these findings, we fit a linear mixed-effects model to sentence ratings for responses to true negative sentences only.  Coefficients for this model can be seen in Table \ref{tab:s1}.  We examined the interaction between context, negation type (e.g. nonexistence or alternative), and negation framing (e.g. ``has no X'' vs ``doesn't have X'').\footnote{ The model specification was as follows: \texttt{rating $\sim$ context~$\times$~negation type~$\times$~negation type + (negation type~$\times$~negation frame~\textbar~subject) +  (negation type~$\times$~negation frame~\textbar~item)}} We found a main effect of context, with true negative sentences presented in a strong context eliciting significantly higher ratings than true negative sentences presented in a weak context ($\beta= .99$, $p< .001$).  We also found main effects of negation type, with sentences referring to an alternative receiving lower ratings than sentences expressing nonexistence ($\beta= -.46$, $p< .05$), as well as negation framing, with sentences of the form ``has no X'' receiving lower ratings than sentences of the form ``doesn't have X''  ($\beta= -.46$, $p< .05$).  There were no interactions between negation frame, negation type, and context.  

The difference between the two negation sentence frames is likely due to frequency effects in the input (GET SOME DATA). 

The fact that negative sentences expressing nonexistence were rated higher than negative sentences referring to an alternative property reflects previous findings from eye-tracking experiments testing the same difference \cite{nordmeyer2014}.  In that study, adults were marginally faster to look at the target of a negative sentence when that sentence referred to nonexistence compared to when the sentence referred to an alternative property.  These data suggest that this finding is not merely due to superficial differences in between the stimuli (e.g. it is easier to identify a character with nothing than a character with a different object), but perhaps because the nonexistence condition was more felicitous than the alternative condition.  (((NOTE: Is this too much of a stretch?))) One explanation for this finding is that negative sentences that refer to an alternative are more informative than negative sentences that refer to nonexistence.  Even in a strong context (e.g. one where all of the context characters have an apple), the sentence ``Abby doesn't have an apple'' is less informative when Abby has a different object (e.g. a cat), because there is a more informative alternative (e.g. ``Abby has a cat'').  When Abby has nothing, however, the sentence ``Abby doesn't have an apple'' is reasonably informative in a context where everyone else has apples, because it uniquely identifies Abby and there isn't a more informative alternative available.  

((NEED A BETTER TRANSITION TO STUDY 2))


\begin{figure}
\begin{center} 
\includegraphics[width=3.25in]{figures/study1.pdf}
\caption{\label{fig:s1} Ratings for different types of true negative sentences in Study 1.  Sentences of the form ``...has no X'' are shown on the left, and sentences of the form ``doesn't have X'' are shown on the right.  Negative sentences expressing nonexistence are shown in black, and negative sentences referring to an alternative object are shown in gray.  Error bars show 95\% confidence intervals.}
\end{center} 
\end{figure}

\begin{table}[t]
\caption{\label{tab:s1} Coefficient estimates from a mixed-effects model predicting ratings for true negative sentences in Study 1.}
\begin{center}
\small\addtolength{\tabcolsep}{-5pt}
\begin{tabular}{rrrr}
  \hline
 & Coefficient & Std. err. & t value \\ 
  \hline
(Intercept) & 5.24 & 0.20 & 25.94 \\ 
  Context (strong) & 0.99 & 0.29 & 3.46  \\ 
  Negation type (alternative) & -0.46 & 0.18 & -2.51 \\
  Frame (``has no'') & -0.51 & 0.25 & -2.01 \\ 
  Context $\times$Negation type & -0.32 & 0.24 & -1.32 \\
  Context $\times$Frame & -0.18 & 0.35 & -0.50 \\
  Negation type$\times$Frame & -0.19 & 0.22 & -0.83 \\
  Context$\times$Negation type$\times$Frame & 0.20 & 0.33 & 0.62 \\
   \hline
\end{tabular}
\vspace{-1.5cm}
\end{center}
\end{table}

\section{Study 2:}

Study 1 found clear differences in participants' ratings of negative sentences.  Participants preferred negative sentences with the sentential framing ``doesn't have X'' compared to the lexical framing ``has no X''.  Participants also preferred negative sentences that referred to nonexistence (e.g. a negative sentence that refers to the complete absence of an object, rather than an alternative object).  Across both of these factors, negative sentences that were presented within a strong context (where everyone else in the context possessed the negated object) received higher ratings than negative sentences presented in a weak context, where none of the characters in the context possessed any object.

In the next study, we further explore this context effect by focusing specifically on negation-as-alternative sentences with a ``doesn't have'' framing.  In Study 2, context was a within-subjects factor, so that participants saw multiple items of different context types throughout the experiment.  We examined the effect of three different types of context: a ``target'' context, in which all context characters had the negated target object (identical to the strong context in Study 1), a ``none'' context, in which none of the context characters had any objects (identical to the weak context in Study 1), and a ``foil'' context, in which all context characters had an alternative object (e.g. a different object than the one negated in the negative sentences). 

In Study 2, we expected to replicate the same difference between the none context and the target context as was seen in Study 1: Negative sentences presented in a target context should receive higher ratings than negative sentences presented in a none context.  We had multiple hypotheses about the foil context.  In true negative trials with a foil context, all characters (including the target character) had the same objects on their table (e.g. cats), and the negative sentence referred to a different object (e.g. ``X doesn't have apples'').  Some previous work has suggested that a critical element of the effect of context on negative sentences is the fact that the referent of the negative sentence is the ``odd one out'' \cite{wason1965}.  If this is the case, the foil context might be even worse than the none context, because the target of the negative sentence does not stand out from the context.  If, however, the mechanism by which context influences negation is the fact that context changes the informativeness of negation, then there should be no difference between the foil and none contexts, because negative sentences are no less informative in the foil context.  

\subsection{Method}

\subsubsection{Participants}
We recruited 194 participants to participate in an online experiment through the Amazon's Mechanical Turk (mTurk) website.  Participants ranged in age from 18-65; 115 male and 76 female (3 declined to report gender).  We restricted participation to individuals in the United States. We paid participants 40 cents to participate, which took approximately 7 minutes to complete.  


\subsubsection{Stimuli}
Trials in Study 2 had the same structure as trials in Study 1, with the following exceptions:

We created 24 trials for Study 2.  All negative sentences were of the form ``[CHARACTER] doesn't have a/an [TARGET OBJECT]''.  On each trial, the target character either had a target item on their table, or had an alternative item (eliminating the negation-as-nonexistence trials).  Each participant evaluated nine true positive sentences, three false positive, three false negative, and nine true negative trials.

In Study 2, context was a within-subjects factor with three levels.  Trials either had no context (e.g. context characters had nothing on their table, identical to the no context condition in Study 1), target context (i.e. context characters each had a target item on their table, same as the context condition in Study 1), or a foil context (i.e. context characters had an alternative item on their table, e.g. all characters have cats, and the sentence is about the presence/absence of apples).  Each context condition appeared an equal number of times within each trial type.  

\subsubsection{Procedure}
The procedure was identical to Study 1.

\subsubsection{Data Processing}
We excluded from analysis four participants who did not list English as their native language, six participants for having participated in a previous pilot study, and 34 participants for using fewer than 3 points on the scale.  Thus, data from a total of 154 participants were analyzed.  

\subsection{Results \& Discussion}
True negative sentences were rated significantly higher when they were presented in a strong target context compared to either the none context or the foil context (see Figure \ref{fig:modelvdata}).  There was no difference between sentences presented in the none context condition vs the foil context condition, nor was there any effect of context on true positive sentences or false sentences.  

We fit a linear mixed-effects model to sentence ratings to test the interaction between context, sentence type, and truth value.\footnote{ The model specification was as follows: \texttt{rating $\sim$ context~$\times$~sentence type~$\times$~truth value + (sentence type~$\times$~truth value~\textbar~subject) +  (sentence type~$\times$~truth value~\textbar~item)}}  Coefficients for this model can be seen in Table \ref{tab:s2}.  We found a significant effect of truth value, with true sentences receiving significantly higher ratings than false sentences, ($\beta= 5.01$, $p< .001$).  A significant interaction between sentence type and truth value indicates that true negative sentences received lower ratings than true positive sentences, ($\beta= -1.78$, $p< .001$).  Finally, a three-way interaction between between sentence type, truth value, and context was significant for the target context ($\beta= 1.08$, $p< .001$), and marginally significant for the foil context ($\beta= .34$, $p=.06$), indicating that true negative sentences received significantly higher ratings in the target context compared to the none context, and slightly higher ratings in the foil context compared to the none context.  


\begin{table}[t]
\caption{\label{tab:s2} Coefficient estimates from a mixed-effects model predicting sentence ratings in Study 2.}
\begin{center}
\small\addtolength{\tabcolsep}{-5pt}
\begin{tabular}{rrrr}
  \hline
 & Coefficient & Std. err. & t value \\ 
  \hline
(Intercept) & 1.62 & 0.11 & 15.07 \\ 
  Context (foil) & 0.13 & 0.11 & 1.14  \\ 
  Context (target) & 0.05 & 0.11 & 0.43  \\ 
  Sentence type (negative) & 0.11 & 0.12 & 0.94 \\
  Truth value (true) & 5.01 & 0.13 & 37.60 \\ 
  Context (foil)$\times$Sentence & -0.16 & 0.16 & -0.98 \\
  Context (target)$\times$Sentence & -0.11 & 0.16 & -0.72 \\
  Context (foil)$\times$Truth & -0.19 & 0.13 & -1.48 \\
  Context (target)$\times$Truth & -0.14 & 0.13 & -1.08 \\
  Sentence$\times$Truth & -1.78 & 0.16 & -11.40 \\
  Context (foil)$\times$Sentence$\times$Truth& 0.34 & 0.18 & 1.87 \\
  Context (target)$\times$Sentence$\times$Truth & 1.08 & 0.18 & 5.88 \\
   \hline
\end{tabular}
\vspace{-1.5cm}
\end{center}
\end{table}

DISCUSSION HERE


\section{Model}

Studies 1 and 2 explored the effects of context on different types of negative sentences.  Both studies found a significant effect of context on participants' ratings of true negative sentences.  Why does context have this effect on negative sentences?  One possibility is that felicity ratings are influenced by the \emph{informativeness} of negative sentences.  According to \citeNP{grice1975}, speakers should produce sentences that are appropriately informative based on the context.  In a context where most characters have apples and another character has a cat, it is informative to mention the ``odd character out'' by describing the cat, because this feature is unique to the character being described.

Previous work suggests that listeners expect speakers to produce informative utterances \cite{frank2012}, and that this can predict adults' speed to evaluate negative sentences \cite{nordmeyer2013}.  Here, we use the same model to explore whether participants' felicity judgments can be predicted by the informativeness of those sentences in context.  We focus on true negative sentences here, because the ceiling effect for true positive sentences and the floor effect for negative sentences make it difficult to consider the effect of context on these sentences.  

((NOTE: This is basically just from last year's paper.  Is it even worth walking in detail through the model here, or should I describe it more generally and refer readers to last year's paper?))
We modeled the behavior of participants in our experiments by assuming that participants' felicity judgments are proportional to the probability of the utterance $w$, given the context $C$ and the speaker's intended referent $r_S$:

\begin{equation}\label{eq:surprise}
Rating \sim -P(w| r_s, C).
\end{equation}

\noindent We then define the probability of the utterance as proportional to its utility (following \citeNP{frank2012}):

\begin{equation}\label{eq:pw1}
P(w | r_s, C) \propto  e^{U(w;r_s,C)},
\end{equation} 

\noindent This utility is defined as the informativeness of $w$ minus its cost $D(w)$:

\begin{equation}\label{eq:utility}
U(w;r_s,C) = I(w;r_s, C) - D(w).
\end{equation}

\noindent Informativeness in context is calculated as the number of bits of information conveyed by the word. We assume that $w$ has a uniform probability distribution over its extension in context (e.g.\ ``has apples'' applies to any character who has apples, leading to a probability of $1/|w|$ of picking out each individual character with apples) :

\begin{equation}\label{eq:info}
I(w;r_s, C) = -(-\log(|w|^{-1})).
\end{equation}

\noindent The cost term $D(w)$ can be defined in any number of ways; in this model we define it as the number of words in the utterance multiplied by a cost-per-word parameter.  In our simulations, we did not differentiate between different negative sentence frames, and treat negative sentences as having one word more than positive sentences.  ((NOTE: I'm not even showing positive sentences here, so I should probably just cut this))

We created a sparse vocabulary which represented possible words to describe the characters.  This included the target utterance (e.g.\ ``has apples'' and ``doesn't have apples''), an utterance referring to the alternative item (e.g. ``has cats''/``doesn't have cats''), as well as words that were uniformly true or false of all characters. Combining Equations \ref{eq:pw1}--\ref{eq:info}, and normalizing Eq.\ \ref{eq:pw1} over all possible words in the vocabulary $V$, we have:

\begin{equation}\label{eq:pw2}
P(w | r_s, C) = \frac{ e^{\log(|w|^{-1}) - D(w)}} {\sum_{w' \in V}{e^{\log(|w'|^{-1}) - D(w')}}}.
\end{equation}

\noindent Combining Eq. \ref{eq:surprise} with Eq. \ref{eq:pw2}, this model predicts that as the number of characters with target items in the context increases, the informativeness of a sentence negating the target item decreases, because it selects an increasingly smaller subset of the context. Highly informative sentences will have high probability. 

\begin{figure}
\begin{center} 
\includegraphics[width=3.25in]{figures/modelcomp.pdf}
\caption{\label{fig:modelvdata} CAPTION HERE.}
\end{center} 
\end{figure}



\section{General Discussion}

\bibliographystyle{apacite}

\setlength{\bibleftmargin}{.125in}
\setlength{\bibindent}{-\bibleftmargin}

\bibliography{bibLibrary}


\end{document}
